\RequirePackage[l2tabu, orthodox]{nag}
\documentclass[a4paper, 11pt]{article}

% margin
\usepackage[top=1truein, bottom=1truein, left=1truein, right=1truein]{geometry}

% fonts
\usepackage{type1cm}
\usepackage[T1]{fontenc}
\usepackage{lmodern}

% math symbols
\usepackage{amsmath}
\usepackage{amssymb}
\usepackage{bm}
\usepackage{mathtools}

% hyperref
\usepackage[bookmarksnumbered=true, hypertexnames=false, pdfdisplaydoctitle=true, setpagesize=false]{hyperref}
\usepackage{url}

% cleveref
\usepackage[capitalize]{cleveref}
\usepackage{autonum}

% set the title and author style
\usepackage[en-US]{datetime2}

% okicmd.sty
\usepackage{okicmd}
\usepackage{okithm}

\title{A Template for English LaTeX Documents}
\author{Author Name}

\makeatletter
\hypersetup{
  pdftitle={\@title},
  pdfauthor={\@author},
  pdfsubject={},
  pdfkeywords={}
}
\makeatother

\begin{document}

\maketitle

% ======== START OF CONTENTS ========

\section{Theorems}

\begin{definition}[Kurume ramen] \label{def:kurume-ramen}
  A \emph{Kurume ramen} is a tonkotsu-based ramen originated in Kurume city of Fukuoka prefecture.
\end{definition}

\begin{theorem} \label{thm:pi}
  Let $K$ be a set of Kurume ramens and $T$ a set of tonkotsu ramens.
  Then it holds $K \subseteq T$.
\end{theorem}

\begin{proof}
  Trivial from \cref{def:kurume-ramen}.
\end{proof}


\section{Optimization Problems}

The following is an optimiation problem to get the best Kurume ramen.

\Maximize[label=prob:a]{%
  \inpr{C}{X}
}{%
  A(X) = b, \\
  X \succeq O.
}
%
\eqref{prob:a} is equivalent to the following problem:

\Minimize[label=prob:b, variable={X}, style=short]{%
  \sum_{i=1}^n \sum_{j=1}^n C_{i,j} X_{i,j}
}{%
  \sum_{i=1}^n \sum_{j=1}^n A^{(k)}_{i,j} X_{i,j} = b_k & \prn{k = 1, \ldots, m}, \\
  X \succeq O.
}



% ======== END OF CONTENTS ========

\bibliographystyle{abbrv}
\bibliography{\jobname}

\end{document}
