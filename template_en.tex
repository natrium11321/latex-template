\RequirePackage[l2tabu, orthodox]{nag}
\documentclass[a4paper, 11pt]{article}

% margin
\usepackage[top=1truein, bottom=1truein, left=1truein, right=1truein]{geometry}

% fonts
\usepackage{type1cm}
\usepackage[T1]{fontenc}
\usepackage{lmodern}

% math symbols
\usepackage{amsmath}
\usepackage{amssymb}
\usepackage{bm}
\usepackage{mathtools}

% hyperref
\usepackage[bookmarksnumbered=true, hypertexnames=false, pdfdisplaydoctitle=true, setpagesize=false]{hyperref}
\usepackage{url}

% cleveref
\usepackage[capitalize]{cleveref}
\usepackage{autonum}

% set the title and author style
\usepackage[en-US]{datetime2}

% okicmd.sty
\usepackage{okicmd}
\usepackage{okithm}


\title{A Template for English LaTeX Documents}
\author{Author Name}

\makeatletter
\hypersetup{
  pdftitle={\@title},
  pdfauthor={\@author},
  pdfsubject={},
  pdfkeywords={}
}
\makeatother

\begin{document}

\maketitle

% ======== START OF CONTENTS ========

\section{Theorems}

\begin{theorem} \label{def:kurume-ramen}
  A \emph{Kurume ramen} is a tonkotsu-based ramen originated in Kurume city of Fukuoka prefecture.
\end{theorem}

\begin{theorem} \label{thm:pi}
  Let $K$ be a set of Kurume ramens and $T$ a set of tonkotsu ramens.
  Then it holds $K \subseteq T$.
\end{theorem}

\begin{proof}
  Trivial from \cref{def:kurume-ramen}.
\end{proof}


% ======== END OF CONTENTS ========

\bibliographystyle{abbrv}
\bibliography{\jobname}

\end{document}
