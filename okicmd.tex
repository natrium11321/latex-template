% !TEX encoding = UTF-8 Unicode
% !TEX root = okicmd.tex
\RequirePackage[l2tabu, orthodox]{nag}
\documentclass[a4paper, 11pt]{article}

% Margin & fonts (delete them to use a provided class style)
\usepackage[top=1truein, bottom=1truein, left=1truein, right=1truein]{geometry}
\usepackage{type1cm}
\usepackage[utf8]{inputenc}
\usepackage[T1]{fontenc}
\usepackage{lmodern}

% hyperref
\usepackage[bookmarksnumbered=true, hypertexnames=false, pdfdisplaydoctitle=true, pdfusetitle, setpagesize=false, unicode]{hyperref}
\usepackage{url}

% Clever references
\usepackage{amsmath}
\usepackage[capitalize, noabbrev]{cleveref}
\newcommand{\crefrangeconjunction}{--}
\usepackage{autonum}

% Misc
\usepackage{ascmac}
\usepackage{booktabs}
\usepackage{comment}
\usepackage[en-US]{datetime2}

% okicmd
\usepackage{okicmd}
\usepackage{okithm}

\title{The \texttt{okicmd} and \texttt{okithm} Packages}
\author{Taihei Oki}

\begin{document}
\maketitle

\section{The \texttt{okicmd} Package}

\subsection{Alphabets}
\begin{center}
  \begin{tabular}{lc} \toprule
    \multicolumn{1}{c}{Input} & Output        \\\midrule
    \verb|l|    & $l$           \\
    \verb|\ell|    & $\ell$        \\
    \verb|\epsilon|    & $\epsilon$    \\
    \verb|\varepsilon|    & $\varepsilon$ \\
    \verb|\phi|    & $\phi$        \\
    \verb|\varphi|    & $\varphi$     \\
    \bottomrule
  \end{tabular}
\end{center}

\subsection{Parentheses}
\begin{center}
  \begin{tabular}{lc} \toprule
    \multicolumn{1}{c}{Input} & Output               \\\midrule
    \verb|\prn{\cdot}|    & $\prn{\cdot}$        \\
    \verb|\prn[\big]{\cdot}|    & $\prn[\big]{\cdot}$  \\
    \verb|\prn[\Big]{\cdot}|    & $\prn[\Big]{\cdot}$  \\
    \verb|\prn[\bigg]{\cdot}|   & $\prn[\bigg]{\cdot}$ \\
    \verb|\prn[\Bigg]{\cdot}|   & $\prn[\Bigg]{\cdot}$ \\
    \verb|\curl{\cdot}|   & $\curl{\cdot}$       \\
    \verb|\sqbr{\cdot}|   & $\sqbr{\cdot}$       \\
    \verb|\agbr{\cdot}|   & $\agbr{\cdot}$       \\
    \verb|\dbbr{\cdot}|   & $\dbbr{\cdot}$       \\
    \verb|\abs{\cdot}|   & $\abs{\cdot}$        \\
    \verb|\norm{\cdot}|   & $\norm{\cdot}$       \\
    \verb|\floor{\cdot}|   & $\floor{\cdot}$      \\
    \verb|\ceil{\cdot}|   & $\ceil{\cdot}$       \\
    \bottomrule
  \end{tabular}
\end{center}

\subsection{Logic}
\begin{center}
  \begin{tabular}{lc} \toprule
    \multicolumn{1}{c}{Input} & Output        \\\midrule
    \verb|\bigland|   & $\bigland$    \\
    \verb|\biglor|   & $\biglor$     \\
    \verb|a \defeq b|   & $a \defeq b$  \\
    \verb|b \eqdef a|   & $b \eqdef a$  \\
    \verb|P \defiff Q|   & $P \defiff Q$ \\
    \bottomrule
  \end{tabular}
\end{center}

\subsection{Sets}
\begin{center}
  \begin{tabular}{lc} \toprule
    \multicolumn{1}{c}{Input} & Output                   \\\midrule
    \verb|\set{a \in S}|   & $\set{a \in S}$          \\
    \verb|\set{a \in S}[a^2 = 1]|   & $\set{a \in S}[a^2 = 1]$ \\
    \verb|\intset{n}|   & $\intset{n}$             \\
    \verb|\card{X}|   & $\card{X}$               \\
    \verb|\setN|   & $\setN$                  \\
    \verb|\setZ|   & $\setZ$                  \\
    \verb|\setQ|   & $\setQ$                  \\
    \verb|\setR|   & $\setR$                  \\
    \verb|\setC|   & $\setC$                  \\
    \verb|\setH|   & $\setH$                  \\
    \verb|\setF|   & $\setF$                  \\
    \verb|\setK|   & $\setK$                  \\
    \verb|\setZp|   & $\setZp$                 \\
    \verb|\setQp|   & $\setQp$                 \\
    \verb|\setRp|   & $\setRp$                 \\
    \bottomrule
  \end{tabular}
\end{center}

\subsection{Maps}
\begin{center}
  \begin{tabular}{lc} \toprule
    \multicolumn{1}{c}{Input} & Output               \\\midrule
    \verb|\doms{X}{Y}|   & $\doms{X}{Y}$        \\
    \verb|\funcdoms{f}{X}{Y}|   & $\funcdoms{f}{X}{Y}$ \\
    \verb|\restr{f}{S}|   & $\restr{f}{S}$       \\
    \verb|\id_K|   & $\id_K$              \\
    \verb|\dom f|   & $\dom f$             \\
    \verb|\cod f|   & $\cod f$             \\
    \verb|\supp f|   & $\supp f$            \\
    \bottomrule
  \end{tabular}
\end{center}

\subsection{Lattices}
\begin{center}
  \begin{tabular}{lc} \toprule
    \multicolumn{1}{c}{Input} & Output      \\\midrule
    \verb|x \meet y|   & $x \meet y$ \\
    \verb|x \join y|   & $x \join y$ \\
    \verb|\bigmeet|   & $\bigmeet$  \\
    \verb|\bigjoin|   & $\bigjoin$  \\
    \bottomrule
  \end{tabular}
\end{center}

\subsection{Algebra}
\begin{center}
  \begin{tabular}{lc} \toprule
    \multicolumn{1}{c}{Input} & Output               \\\midrule
    \verb|\Hom(G)|   & $\Hom(G)$            \\
    \verb|\End R|   & $\End R$             \\
    \verb|\Aut_k K|   & $\Aut_k K$           \\
    \verb|\gen{a, b}|   & $\gen{a, b}$         \\
    \verb|\gen{a, b}[ab = e]|   & $\gen{a, b}[ab = e]$ \\
    \verb|\abel{G}|   & $\abel{G}$           \\
    \verb|\comm{G}|   & $\comm{G}$           \\
    \verb|\sym_n|   & $\sym_n$             \\
    \verb|\sgn(\sigma)|   & $\sgn(\sigma)$       \\
    \verb|\mult{R}|   & $\mult{R}$           \\
    \verb|\M_{m,n}(R)|   & $\M_{m,n}(R)$        \\
    \verb|\GL_n(R)|   & $\GL_n(R)$           \\
    \verb|\SL_n(R)|   & $\SL_n(R)$           \\
    \verb|\O(n)|   & $\O(n)$              \\
    \verb|\SO(n)|   & $\SO(n)$             \\
    \verb|\U(n)|   & $\U(n)$              \\
    \verb|\SU(n)|   & $\SU(n)$             \\
    \bottomrule
  \end{tabular}
\end{center}

\subsection{Number Theory}
\begin{center}
  \begin{tabular}{lc} \toprule
    \multicolumn{1}{c}{Input} & Output          \\\midrule
    \verb|a \coprime b|   & $a \coprime b$  \\
    \verb|a \divides b|   & $a \divides b$  \\
    \verb|a \ndivides b|   & $a \ndivides b$ \\
    \bottomrule
  \end{tabular}
\end{center}

\subsection{Linear Algebra}
\begin{center}
  \begin{tabular}{lc} \toprule
    \multicolumn{1}{c}{Input} & Output                         \\\midrule
    \verb|\tr A|   & $\tr A$                        \\
    \verb|\rank A|   & $\rank A$                      \\
    \verb|\trank A|   & $\trank A$                     \\
    \verb|\diag(a_1, \ldots, a_n)|   & $\diag(a_1, \ldots, a_n)$      \\
    \verb|\blockdiag(A_1, \ldots, A_n)|   & $\blockdiag(A_1, \ldots, A_n)$ \\
    \verb|\vec(A)|   & $\vec(A)$                      \\
    \verb|\Row(A)|   & $\Row(A)$                      \\
    \verb|\Col(A)|   & $\Col(A)$                      \\
    \verb|\onevec|   & $\onevec$                      \\
    \verb|\trsp{A}|   & $\trsp{A}$                     \\
    \verb|\adjo{A}|   & $\adjo{A}$                     \\
    \verb|\inpr{x}{y}|   & $\inpr{x}{y}$                  \\
    \bottomrule
  \end{tabular}
\end{center}

\subsection{Analysis}
\begin{center}
  \begin{tabular}{lc} \toprule
    \multicolumn{1}{c}{Input} & Output         \\\midrule
    \verb|\e|   & $\e$           \\
    \verb|\d|   & $\d$           \\
    \verb|\dif{f}{x}|   & $\dif{f}{x}$   \\
    \verb|\pdif{f}{x}|   & $\pdif{f}{x}$  \\
    \verb|\ddif{f}{x}|   & $\ddif{f}{x}$  \\
    \verb|\dpdif{f}{x}|   & $\dpdif{f}{x}$ \\
    \bottomrule
  \end{tabular}
\end{center}

\subsection{Complex Analysis}
\begin{center}
  \begin{tabular}{lc} \toprule
    \multicolumn{1}{c}{Input} & Output            \\\midrule
    \verb|\i|   & $\i$              \\
    \verb|\Re z|   & $\Re z$           \\
    \verb|\Im z|   & $\Im z$           \\
    \verb|\Arg z|   & $\Arg z$          \\
    \verb|\Log z|   & $\Log z$          \\
    \verb|\Sin z|   & $\Sin z$          \\
    \verb|\Cos z|   & $\Cos z$          \\
    \verb|\Tan z|   & $\Tan z$          \\
    \verb|\Res_{z=0} f(z)|   & $\Res_{z=0} f(z)$ \\
    \bottomrule
  \end{tabular}
\end{center}

\subsection{Optimization}
\begin{center}
  \begin{tabular}{lc} \toprule
    \multicolumn{1}{c}{Input} & Output                   \\\midrule
    \verb|\argmin_{x \in S} f(x)|   & $\argmin_{x \in S} f(x)$ \\
    \verb|\argmax_{x \in S} f(x)|   & $\argmax_{x \in S} f(x)$ \\
    \verb|\Order(n) |  & $\Order(n)$              \\
    \verb|\order(n) |  & $\order(n)$              \\
    \bottomrule
  \end{tabular}
\end{center}

\section{The \texttt{okithm} Package}
\subsection{Theorems}

If the option \texttt{language = Japanese} is given, \texttt{okithm} will translate all the environment titles (Theorem, Definition, etc.) into Japanese.
You can disable theorems by setting the option \texttt{notheorem}.

\begin{itembox}[l]{Input}
  \begin{verbatim}
\begin{theorem}[Awesome theorem]
  The square root $\sqrt{2}$ of two is irrational.
\end{theorem}
\end{verbatim}
\end{itembox}

\begin{itembox}[l]{Output}
  \begin{theorem}[Awesome theorem]
    The square root $\sqrt{2}$ of two is irrational.
  \end{theorem}
\end{itembox}

\begin{itembox}[l]{Input}
  \begin{verbatim}
\begin{definition}[Coprime]
  Integers $a$ and $b$ are said to be \emph{coprime} if their greatest
  common divisor is one.
\end{definition}
\end{verbatim}
\end{itembox}

\begin{itembox}[l]{Output}
  \begin{definition}[Coprime]
    Integers $a$ and $b$ are said to be \emph{coprime} if their greatest
    common divisor is one.
  \end{definition}
\end{itembox}

\begin{itembox}[l]{Input}
  \begin{verbatim}
\begin{lemma}
  If $a$ and $b$ are coprime, so are $a^2$ and $b^2$.
\end{lemma}
\end{verbatim}
\end{itembox}

\begin{itembox}[l]{Output}
  \begin{lemma}
    If $a$ and $b$ are coprime, so are $a^2$ and $b^2$.
  \end{lemma}
\end{itembox}

\begin{itembox}[l]{Input}
  \begin{verbatim}
\begin{proposition}
  If $\sqrt{2} = a/b$, then $a^2 = 2b^2$.
\end{proposition}
\end{verbatim}
\end{itembox}

\begin{itembox}[l]{Output}
  \begin{proposition}
    If $\sqrt{2} = a/b$, then $a^2 = 2b^2$.
  \end{proposition}
\end{itembox}

\begin{itembox}[l]{Input}
  \begin{verbatim}
\begin{corollary}
  If $\sqrt{2} = a/b$ with $a$ and $b$ being coprime, then $a$ is even.
\end{corollary}
\end{verbatim}
\end{itembox}

\begin{itembox}[l]{Output}
  \begin{corollary}
    If $\sqrt{2} = a/b$ with $a$ and $b$ being coprime, then $a$ is even.
  \end{corollary}
\end{itembox}

\begin{itembox}[l]{Input}
  \begin{verbatim}
\begin{example}
  If $a = 2$ and $b = 1$, then $a$ is even but $\sqrt{2} \ne a/b$.
\end{example}
\end{verbatim}
\end{itembox}

\begin{itembox}[l]{Output}
  \begin{example}
    If $a = 2$ and $b = 1$, then $a$ is even but $\sqrt{2} \ne a/b$.
  \end{example}
\end{itembox}

\begin{itembox}[l]{Input}
  \begin{verbatim}
\begin{remark}
  Note that $a$ and $b$ must be integers.
\end{remark}
\end{verbatim}
\end{itembox}

\begin{itembox}[l]{Output}
  \begin{remark}
    Note that $a$ and $b$ must be integers.
  \end{remark}
\end{itembox}

\begin{itembox}[l]{Input}
  \begin{verbatim}
\begin{proof}
  Suppose to the contrary that $\sqrt{2} = a/b$ with coprime $a$ and $b$.
  Then both $a$ and $b$ are even, which contradicts the assumption.
\end{proof}
\end{verbatim}
\end{itembox}

\begin{itembox}[l]{Output}
  \begin{proof}
    Suppose to the contrary that $\sqrt{2} = a/b$ with coprime $a$ and $b$.
    Then both $a$ and $b$ are even, which contradicts the assumption.
  \end{proof}
\end{itembox}

\subsection{Algorithms}

You can disable algorithms by setting the option \texttt{noalgorithm}.

\begin{itembox}[l]{Input}
  \begin{verbatim}
\begin{algorithm}[htbp]
  \begin{algorithmic}[1]
    \Input{$n \in \setN$}
    \Output{$n(n+1)/2$}
    \State{$s \gets 0$}
    \ForTo{$i = 1$}{$n$}
      \State{$s \gets s + i$}
    \EndFor
    \State{\Return $s$}
  \end{algorithmic}
\end{algorithm}
\end{verbatim}
\end{itembox}

\begin{itembox}[l]{Output}
  \begin{algorithmic}[1]
    \Input{$n \in \setN$}
    \Output{$n(n+1)/2$}
    \State{$s \gets 0$}
    \ForTo{$i = 1$}{$n$}
    \State{$s \gets s + i$}
    \EndFor{}
    \State{\Return{} $s$}
  \end{algorithmic}
\end{itembox}

\subsection{Optimization Problems}

You can change \texttt{minimize}, \texttt{maximize} and \texttt{subject to} into \texttt{min}, \texttt{max} and \texttt{s.t.}, respectively, by setting the option \texttt{optstyle=short}.

\begin{itembox}[l]{Input}
  \begin{verbatim}
\Minimize[name={(P)}]{
  \sum_{\condit{x \in S}[x^2 = 1]} w(x) + \sum_{i=1}^n (p_i + q_i)
}{
  S \subseteq V, \\
  p_i \ge 0 & (i = 1, \ldots, n), \\
  q_i \ge 0 & (i = 1, \ldots, n)
}
\end{verbatim}
\end{itembox}

\begin{itembox}[l]{Output}
  \Minimize[name={(P)}]{
    \sum_{\condit{x \in{} S}[x^2 = 1]} w (x) + \sum_{i=1}^n (p_i + q_i)
  }{
    S \subseteq{} V, \\
    p_i \ge{} 0 & (i = 1, \ldots, n), \\
    q_i \ge{} 0 & (i = 1, \ldots, n)
  }
\end{itembox}

\end{document}
